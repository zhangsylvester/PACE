\documentclass[12pt]{amsart}
\usepackage{adjustbox}

%packagrs

\usepackage{ifthen}
\usepackage[utf8]{inputenc}
\usepackage{setspace}
\usepackage[english]{babel}
\usepackage{enumerate}
\usepackage{thmtools}
\usepackage[shortlabels]{enumitem}
\usepackage[T1]{fontenc}
\usepackage{tikz}
\usetikzlibrary{shapes.geometric,intersections,decorations.markings,snakes}
\usetikzlibrary{calc,intersections,through,backgrounds}
\usetikzlibrary{patterns}
%\usetikzlibrary{pathreplacing}
\usepackage{tkz-graph}
\let\SO\undefined
\usetikzlibrary{arrows}
\usetikzlibrary{decorations.markings}
\usetikzlibrary{automata,positioning}

\renewcommand{\l}[2]{\lambda_{#1#2}}
\newcommand{\bs}[1]{\boldsymbol{#1}}
\newcommand{\tei}[0]{Teichm\"uller}
\newcommand{\calC}[0]{\mathcal{C}}
\newcommand{\calA}[0]{\mathcal{A}}


\DeclareMathOperator{\Hom}{Hom}
\DeclareMathOperator{\psl}{PSL}
\DeclareMathOperator{\rr}{\mathbb{R}}
\DeclareMathOperator{\osp}{Osp}
\DeclareMathOperator{\wt}{wt}
\DeclareMathOperator{\twt}{twt}
\DeclareMathOperator{\crs}{cross}
\DeclareMathOperator{\ber}{Ber}
\DeclareMathOperator{\st}{st}
\DeclareMathOperator{\tp}{t}
\DeclareMathOperator{\zz}{\mathbb{Z}}

\newcommand{\mb}[1]{\mathbb{#1}}
\newcommand{\tld}[1]{\widetilde{#1}}
%
\newcommand{\precdot}{\prec\mathrel{\mkern-5mu}\mathrel{\cdot}}
\makeatletter
\newcommand{\preceqdot}{\mathrel{\mathpalette\pr@ceqd@t\relax}}
\newcommand{\pr@ceqd@t}[2]{%
  \begingroup
  \sbox\z@{$#1\prec$}\sbox\tw@{$#1\preceq$}%
  \dimen@=\dimexpr\ht\tw@-\ht\z@\relax
  {\preceq}%
  \mkern-5mu
  \raisebox{\dimen@}{$\m@th#1\cdot$}%
  \endgroup
}
\makeatother
\usepackage{etoolbox}

\def\SetFancyGraph {
	\SetVertexMath
	\GraphInit[vstyle=Art]
	\SetUpVertex[MinSize=2pt]
	\SetVertexLabel
	\tikzset{VertexStyle/.style = {shape = circle,shading = ball,ball color = black,inner sep = 1.5pt}}
	\SetUpEdge[color=black]
	\tikzset{->-/.style={decoration={ markings, mark=at position 0.8 with {\arrow{>}}},postaction={decorate}}}
	\tikzset{->--/.style={decoration={ markings, mark=at position 0.55 with {\arrow{>}}},postaction={decorate}}}
}


\include{theorems}
\newcommand{\rrr}[0]{\mathscr{R}}
\newcommand{\ttt}[0]{\mathscr{T}}




\newlength\friezelen 
\settowidth{\friezelen}{$\xi_{m}$} % calculate width of widest element

\usepackage{array} % for "\newcolumntype" macro
\newcolumntype{Q}{>{\centering}p{\friezelen}<{}}

\usepackage{mathrsfs} 
\usepackage{mathtools}
\usepackage{verbatim}
\usepackage{comment}
\usetikzlibrary{arrows}
\tikzset {->-/.style={decoration={markings, mark=at position .5 with {\arrow{latex}}}, postaction={decorate}}}
\tikzset {-->-/.style={decoration={markings, mark=at position .5 with {\arrow[scale=2]{latex}}}, postaction={decorate}}}
\newcommand{\midarrow}{\tikz \draw[-triangle 90] (0,0) -- +(.1,0);}
\newcommand{\miduparrow}{\tikz \draw[-triangle 90] (0,0) -- +(.1,0);}
\newcommand{\midrevarrow}{\tikz \draw[-triangle 90] (0,0) -- +(-.1,0);}
\newcommand{\shiftright}[2]{\makebox[#1][r]{\makebox[0pt][l]{#2}}}

\newcommand{\lhdot}[0]{\adjustbox{lap={\width}{0.6em}}{$\cdot$}\lhd}

\usepackage{comment}
\usepackage{graphicx}
\usepackage{color}
\usepackage{etoolbox}
\usepackage[margin=1in]{geometry}
\usepackage{amsmath,amsthm,amssymb,graphicx,tikz,tikz-cd}
\usepackage{hyperref}
\hypersetup{
    colorlinks = true,
    citecolor = orange,%
}
\usepackage[noabbrev]{cleveref}
\usepackage{subcaption}

\makeatletter
\patchcmd{\@settitle}{\uppercasenonmath\@title}{}{}{}
\patchcmd{\@setauthors}{\MakeUppercase}{}{}{}
\patchcmd{\section}{\scshape}{}{}{}
\makeatother
\makeatletter
\patchcmd{\@maketitle}
  {\ifx\@empty\@dedicatory}
  {\ifx\@empty\@date \else {\vskip2ex %vertical position of date
  \centering\footnotesize\@date\par\vskip1ex}\fi
   \ifx\@empty\@dedicatory}
  {}{}
\patchcmd{\@adminfootnotes}
  {\ifx\@empty\@date\else \@footnotetext{\@setdate}\fi}
  {}{}{}
\makeatother

\usepackage{parskip}
\renewcommand{\l}[2]{\lambda_{#1,#2}}

\theoremstyle{plain}
\newtheorem{theorem}{Theorem}[section]
\newtheorem{restate}{Theorem}[section]
\newtheorem{lemma}[theorem]{Lemma}
\newtheorem{prop}[theorem]{Proposition}
\newtheorem{conj}[theorem]{Conjecture}
\newtheorem{question}[theorem]{Question}
\newtheorem{corollary}[theorem]{Corollary}
\newtheorem{claim}[theorem]{Claim}
\theoremstyle{definition}
\newtheorem{remark}[theorem]{Remark}
\newtheorem{example}[theorem]{Example}
\newtheorem{definition}[theorem]{Definition}
\newtheorem{exercise}[theorem]{Exercise}
\newtheorem{problem}[theorem]{Problem}
\DeclareMathOperator{\fA}{\mathfrak{A}}
\DeclareMathOperator{\fC}{\mathfrak{C}}
\DeclareMathOperator{\row}{Row}

\newcommand{\todo}[1]{{\color{red}[To Do: #1]}}
\newcommand{\syl}[1]{{\color{orange}[Sylvester: #1]}}
\newcommand{\new}[1]{{#1}}
\newcommand{\calP}[0]{\mathcal{P}}
\newcommand{\calPn}[0]{\mathcal{P}^{(n)}}
\newcommand{\fc}[1]{\mathfrak{c}_{#1}}
\newcommand{\fa}[1]{\mathfrak{a}_{#1}}
\newcommand{\cal}[1]{\mathcal{#1}}

\title{Problems for PACE 2023}

\author[S. Zhang]{Sylvester W. Zhang}
\date{\today}

\begin{document}
\maketitle


\section{Introduction}

Recall that an \emph{order ideal} of a poset is a downward closed subposet. For $P$ a poset, denote $J(P)$ the set of its order ideals. We consider an action on $J(P)$ called \emph{rowmotion}.

Rowmotion, denoted $\row$, is an invertible operator on the order ideals of any partially ordered set. For an order ideal $I$, $\row(I)$ is the order ideal generated by the minimal elements of the complement of $I$. Rowmotion was first introduced by Brouwer and Schrijver \cite{bs74} and has been extensively studied by many different authors, including \cite{rowfirst,fon1993orbits,toggles,Pan09,prorow}. The name `rowmotion' is due to Striker and Williams \cite{prorow}.

The action of Rowmotion turned out to be extremely nice on certain posets, in particular,  the \emph{rectangle poset} $\mathscr{R}(a,b):=\{(i,j)\in\mathbb Z^2:1\le i\le a,1\le j\le b\}$. Notably, rowmotion on $\mathscr{R}(a,b)$ is a canonical example of the cyclic sieving phenomenon. In particular, it is periodic of order $a+b$, i.e. $\row^{a+b}(I)=I$ for $I\in J(\mathscr{R}(a,b))$.

It has been observed and conjectured (N. Williams) that a seemingly not related poset, the \emph{trapezoid poset} $\mathscr{T}(a,b):=\{(i,j)\in\mathbb Z^2:1\le i\le a,i\le j\le a+b-i\}$ also has periodic orbit under rowmotion.

\begin{figure}[ht]
\begin{center}
\begin{tikzpicture}[xscale=-0.6,yscale=0.6]
	\SetFancyGraph
	\Vertex[NoLabel,x=0,y=0]{1}
	\Vertex[NoLabel,x=-1,y=1]{2}
	\Vertex[NoLabel,x=1,y=1]{3}
	\Vertex[NoLabel,x=-2,y=2]{4}
	\Vertex[NoLabel,x=0,y=2]{5}
	\Vertex[NoLabel,x=2,y=2]{6}
	\Vertex[NoLabel,x=-3,y=3]{7}
	\Vertex[NoLabel,x=-1,y=3]{8}
	\Vertex[NoLabel,x=1,y=3]{9}
	\Vertex[NoLabel,x=-2,y=4]{10}
	\Vertex[NoLabel,x=0,y=4]{11}
	\Vertex[NoLabel,x=-1,y=5]{12}
	\Edges[style={thick}](1,7)
	\Edges[style={thick}](3,10)
	\Edges[style={thick}](6,12)
	\Edges[style={thick}](1,6)
	\Edges[style={thick}](2,9)
	\Edges[style={thick}](4,11)
	\Edges[style={thick}](7,12)
	\draw [decorate,decoration={brace,amplitude=10pt},yshift=7pt] (2.2,1.8)--(-1,5) node [black,midway,yshift=0.5cm,xshift=-0.5cm]  {\rotatebox{45}{\footnotesize $b$}};
	
	
	
	\draw [decorate,decoration={brace,amplitude=10pt,mirror},yshift=-2pt] (2.1,2.1)--(-0.1,-0.1)  node [black,midway,xshift=-0.5cm,yshift=-0.32cm]   {\rotatebox{45}{\footnotesize $a$}};
	
	\node (a) at (0,-2) {$\mathscr{R}(a,b)$};
\end{tikzpicture}
\quad\quad\quad\quad\quad\quad\quad
\begin{tikzpicture}[scale=0.57]
	\SetFancyGraph
	\begin{scope}[yscale=-1,xscale=1]
	\Vertex[NoLabel,x=0,y=0]{1}
	\Vertex[NoLabel,x=2,y=0]{2}
	\Vertex[NoLabel,x=4,y=0]{3}
	\Vertex[NoLabel,x=-1,y=1]{4}
	\Vertex[NoLabel,x=1,y=1]{5}
	\Vertex[NoLabel,x=3,y=1]{6}
	\Vertex[NoLabel,x=0,y=2]{7}
	\Vertex[NoLabel,x=2,y=2]{8}
	\Vertex[NoLabel,x=-1,y=3]{9}
	\Vertex[NoLabel,x=1,y=3]{10}
	\Vertex[NoLabel,x=0,y=4]{11}
	\Vertex[NoLabel,x=-1,y=5]{12}
	\Edges[style={thick}](1,4)
	\Edges[style={thick}](1,5)
	\Edges[style={thick}](2,5)
	\Edges[style={thick}](2,6)
	\Edges[style={thick}](3,6)
	\Edges[style={thick}](4,7)
	\Edges[style={thick}](5,7)
	\Edges[style={thick}](5,8)
	\Edges[style={thick}](6,8)
	\Edges[style={thick}](7,9)
	\Edges[style={thick}](7,10)
	\Edges[style={thick}](8,10)
	\Edges[style={thick}](9,11)
	\Edges[style={thick}](10,11)
	\Edges[style={thick}](11,12)
	\draw [decorate,decoration={brace,amplitude=10pt},yshift=7pt]   (4.2,-0.2)--(-1,5) node [black,midway,yshift=-0.7cm,xshift=0.9cm]  {\rotatebox{45}{\footnotesize $a+b-1$}};
	
	\draw [decorate,decoration={brace,amplitude=10pt,mirror},yshift=-2pt] (4.1,-0.1)--(-0.1,-0.1) node [black,midway,yshift=0.6cm]  {\footnotesize $a$};
	\end{scope}
	\node (a) at (1.8,-6.7) {$\mathscr{T}(a,b)$};
\end{tikzpicture}
\end{center}

    \caption{The rectangle poset {$\mathscr R(3,4)$} and trapezoid poset {$\mathscr{T}(3,4)$}.}
    \label{fig:rect_trap_normal}
\end{figure}

Hopkins \cite{hopkins2019minuscule} conjectured, and Dao, Wellman, Yost--Wolff and myself \cite{dao2020rowmotion} proved that, $\row$ and $J(\rrr(a,b))$ and $J(\ttt(a,b))$ have the same orbit structure. Hopkins further conjectured that, rowmotion on $P$-partitions (generalization of order ideals) of $\rrr$ and $\ttt$ also have the same orbit structure. 

In this project we propose a conjectural bijection to attack Hopkin's conjecture, but before that, we first define $P$-partitions and rowmotion on them.
\section{More on Rowmotion}
As we briefly mentioned in the introduction, the original definition of rowmotion is defined as follows. Take $I\in J(P)$ an order ideal, and find the minimal non-elements of $I$, then $\row(I)$ is the order ideal generated by (below) those elements. See \Cref{fig:rowmotion_example} for example.
\begin{figure}[h]
    \centering
    \begin{tikzpicture}[scale=0.65,rotate=45]
	\foreach \i in {0,...,4}{
    \foreach \j in {0,1,2}
    	\filldraw (\i,\j) circle (1.6pt);}
    \foreach \t in {0,1,2}{
    \draw [-] (0,\t)--(4,\t);
    }
    \foreach \t in {0,...,4}{
    \draw [-] (\t,0)--(\t,2);
    }
    
    \foreach \i in {0,1,2,3}{
    \filldraw [red] (\i,0) circle (2.0 pt);
    }
    \foreach \i in {0,1,2}{
    \filldraw [red] (\i,1) circle (2.0 pt);
    }
    \foreach \i in {0}{
    \filldraw [red] (\i,2) circle (2.0 pt);
    }
    \begin{scope}[xshift=180,yshift=-180]
    \foreach \i in {0,...,4}{
    \foreach \j in {0,1,2}
    	\filldraw (\i,\j) circle (1.6pt);}
    \foreach \t in {0,1,2}{
    \draw [-] (0,\t)--(4,\t);
    }
    \foreach \t in {0,...,4}{
    \draw [-] (\t,0)--(\t,2);
    }
    
    \foreach \i in {0,1,2,3,4}{
    \filldraw [red] (\i,0) circle (2.0 pt);
    }
    \foreach \i in {0,1,2,3}{
    \filldraw [red] (\i,1) circle (2.0 pt);
    }
    \foreach \i in {0,1}{
    \filldraw [red] (\i,2) circle (2.0 pt);
    }
    
    \end{scope}
    \draw[->] (3.8,-0.8) to node [above] {$\row$} (6.5,-5+1.5);


\end{tikzpicture}
    \caption{Rowmotion acting on order ideals.}
    \label{fig:rowmotion_example}
\end{figure}

From this definition it is unclear whether $\row$ is invertible or not.
It become clear, however, from the following alternative definition of
Cameron and Fon-der-Flaass \cite{toggles}.

\begin{prop}\label{prop:row_as_toggles}
	For $p\in\mathcal P$ and order ideal $I$, we define the toggle operation of $p$ on $I$ as follows.
\[\tau_p(I)=\begin{cases}
I\cup p&\text{if }p\notin I\text{ and } I\cup p\in J(\mathcal P)\text{,}\\
I\setminus p&\text{if }p\in I\text{ and }I\setminus p\in J(\mathcal P)\text{,}\\
I&\text{otherwise.}
\end{cases}\]
Then rowmotion is performed by toggles ``row by row'' %\footnote{Here `row' refers to a rank of a poset, which is not a row but a diagonal in a Young diagram notation.} 
from the largest to smallest, i.e.
$$\row(I) =\tau_{p_1}\circ\tau_{p_{n-1}} \circ\cdots\circ\tau_{p_n}(I)$$
where $p_1\le\cdots\le p_n$ is any linear extension of the poset $\mathcal P$.
\end{prop}
\begin{exercise}
	Prove the above proposition, i.e. show that the toggle definition of $\row$ agrees with the original definition.
\end{exercise}
\begin{exercise}
	Identify the orbits of $\row$ on the rectangle poset $\rrr(2,2)$ and the trapezoid poset $\ttt(2,2)$. And speculate a bijection which commutes with rowmotion.
\end{exercise}
\section{$P$-Partitions.} 



For a poset $P$, a $P$-partition is an order preserving map $\varphi$ from $P$ to $\mathbb{N}$, i.e. 
\[a\leq_P b \iff \varphi(a)\leq \varphi(b) \]
They can be thought of as increasing labelings of the poset. A $P$-partition is said to have \emph{height} if all the labelings are smaller than or equal to $m$. We denote $\cal P^m(P)$ the set of all $P$-partitions of height $m$. Note that $J(P)=\cal P^1(P)$.

Stanley defined the \emph{order polynomial} of a poset as follows
\[\Omega_P(m)=\# \calP^{m-1}(m).\]

For rectangle posets, MacMahon (1915) has a very nice formula
\[\Omega_{\rrr(a,b)}(m)=\prod_{i=1}^a\prod_{i=1}^b{i+j+m-2\over i+j-1}\]

And furthermore, Proctor (1983) proved that, 
\[\Omega_{\rrr(a,b)}(m) = \Omega_{\ttt(a,b)}(m)\]
suggesting that $P$-partitions of $\rrr$ are equinumerous with those of $\ttt$. Proctor's proof was not bijective, and the first bijective proof is due to \cite{HPPW18}, using $K$-theoretic tableaux combinatorics.
\section{Rowmotion on $P$-partitions}

There is a natural translation for rowmotion on the level of $P$-partitions, using \emph{piecewise linear toggles}.
\begin{definition}
	For $p\in P$  and $\varphi:P\to \mathbb{N\leq m}$ and $P$-partition of height $m$, we define the piecewise-linear toggle operation of $p$ on $\varphi$ as follows. First extend the poset $P$ to $\hat P=P\cup\{\hat 0,\hat 1\}$ where $\hat 0$ is below everything and $\hat 1$ is above everything, and set $\varphi(\hat 0)=0$ and $\varphi(\hat 1)=m$. Then,
	\[\tau_p(\varphi)(t) := \begin{cases}
		t&\text{ if }t\neq p\\
		\min\{\varphi(i):p\lessdot i\} +\max\{\varphi(i):i\lessdot p\}-\varphi(p)&\text{ if }t = p
	\end{cases}\]
	We then define piecewise linear rowmotion in a similar way.
	$$\row_{pl}(\varphi) =\tau_{p_1}\circ\tau_{p_{n-1}} \circ\cdots\circ\tau_{p_n}(\varphi)$$
where $p_1\le\cdots\le p_n$ is any linear extension of the poset $\mathcal P$.
\end{definition}
\begin{exercise}
	(1) Show that the piecewise linear toggle is a well-defined action on $P$-partitions. In other words, $\tau_p(\varphi)$ will still be a $P$-partition.
	
	(2) Show that when height $m=1$, piecewise linear rowmotion agrees with order ideal rowmotion.
\end{exercise}

\begin{exercise}
	Identify the orbits of rowmotion on $\cal P^3(\rrr(2,2))$ and $\cal P^3(\ttt(2,2))$ and speculate a bijection.
\end{exercise}


Hopkins \cite{hopkins2019minuscule} further conjectured that $\rrr$ and $\ttt$ also have the same rowmotion orbit structure on $P$-partitions. This suggests that there should be a bijection between $\cal P^m(\rrr)$ and $\cal P^m(\ttt)$ equivarient with respect to rowmotion. The \cite{HPPW18} bijection, however, only works in the case of $m=1$ (result of \cite{dao2020rowmotion}).

\begin{problem}\label{prob1}
	Find a bijection between $\cal P^m(\rrr(a,b))$ and $\cal P^m(\ttt(a,b))$ which commutes with piecewise linear rowmotion. In particular, we have a conjectural bijection which we know to commute with rowmotion, but we aren't able to show that it's actually a bijection.
\end{problem}
\section{Birational Rowmotion}
One can de-tropicalize the notion of piecewise linear toggle, and lift it to a birational map. Precisely, for any labelling $\varphi$ of the poset, define birational toggling to be
	\[\tau_p(\varphi)(t) := \begin{cases}
		\varphi(t)&\text{ if }t\neq p\\
		{\displaystyle {1\over\varphi(p)}{\sum_{p\lessdot i}\varphi(i)\over \sum_{i\lessdot p}{1\over \varphi(i)}}}
		&\text{ if }t = p
	\end{cases}\]
	We can then define birational rowmotion in a similar maner.
Grinberg-Roby conjectured that birational rowmotion on the trapezoid poset $\ttt(a,b)$ is periodic of $a+b$. 

\begin{problem}
	Does the conjectural bijection in \Cref{prob1} lift to a bijective map which is equivarient with respect to birational rowmotion?
\end{problem}






\bibliographystyle{alpha}
\bibliography{main.bib}

\end{document}